\documentclass[11pt]{article}
\usepackage[utf8]{inputenc}
\usepackage{amsmath,amsfonts,amssymb}
\usepackage[amsthm]{ntheorem} %amsthm but also no newline in lists!
\usepackage[a4paper,margin=2.5cm]{geometry} %greater control over layout
\usepackage{tikz, pgf, pgfplots} %allows tikzpictures
\pgfplotsset{compat=1.15}
\usepackage{mathrsfs}
\usetikzlibrary{arrows}
\usetikzlibrary{matrix}
\usepackage{url} %urls easy
\usepackage[shortlabels]{enumitem} %allows greater customisability of enum
\usepackage{textcomp} %gets rid of not defining \perthousand and \micro somehow
\usepackage{gensymb} %symbol package
\usepackage{sectsty}
\usepackage{tabto} %absolute/relative horizontal jumps in text
\usepackage{perpage} %footnote counter resets each page
\usepackage[symbol]{footmisc} %use symbols instead of numbers for footnotes
\usepackage{float} %allows "H" parameter for images
\usepackage[htt]{hyphenat} %tt text breaks properly
\usepackage{soul} %strikethrough text available

\begin{document}
	\pagenumbering{gobble}
		\noindent \Large\textbf{Mathematical Olympiads Discord Server}
		\vspace{5pt}\\
		\noindent \huge\textbf{2019 August Intermediate Contest}\\
		\noindent \makebox[\linewidth]{\rule{\textwidth}{0.4pt}}
			
	\normalsize
	
	\begin{flushright}
	\textit{Time: 4 hours} \hfill \textit{Each problem is worth 7 points}
	\end{flushright}
	
	\noindent \textit{Calculators and protractors are not allowed. Do not write your name on your working. After your timeslot finishes, please read the instructions in} \texttt{\#how-to-submit-scripts}\textit{. Do not discuss the contents of this paper outside the text channel }\texttt{\#finished-contestants}\textit{ and the voice channel }\texttt{Post-Contest Banter}\textit{ until notified by staff.}
	
	\paragraph{Problem \large \(\times\).} Find the smallest positive integer \(c\) such that for any positive integer $n$, there exist integers \(m\) and \(k>1\) satisfying \[1! \times 2! \times \cdots \times (cn)! = m! \times k^2.\]
	
	\paragraph{Problem \(\square\).} Consider a \(20\) by \(19\) grid with each square coloured black or white. Andy the ant starts at the center of the square at the bottom-left corner of the grid and travels in horizontal or vertical line segments of positive integer length, never travelling outside the grid. Each time he crosses a black-white boundary, he gets a point. Andy cannot visit a square twice, and before starting he must change the colour of exactly half the squares in the grid.
	
	What is the largest number of points Andy can obtain, regardless of how the grid is initially coloured?

	\paragraph{Problem {\footnotesize \raisebox{1.6pt}{\(\bigcirc\)}}.} \(ABC\) is a triangle and \(P\) is a variable point on side \(BC\) distinct from \(B\) and \(C.\) Let \(I_b\) and \(I_c\) be the incentres of \(ABP\) and \(ACP\) respectively. Prove that the circumcircle of \(PI_bI_c\) passes through a fixed point as \(P\) varies over \(BC.\)

	\paragraph{Problem \(\triangle\).}
	Let \(P\) be a subset of points in the plane, and let \(S\) be a subset of the real numbers. Let a function \(f : P \rightarrow S\) be called \emph{silly} if for any point \(D\) strictly inside any triangle \(ABC\), we have \[f(D) < \frac{f(A) + f(B) + f(C)}{3}.\]
	\begin{enumerate}[(a)]
	\item Let \(\mathbb{Z}\) denote the set of integers, and \(\mathbb{Z}^2\) denote the set of all points with integer coordinates. Does there exist a silly function \(f : \mathbb{Z}^2 \rightarrow \mathbb{Z}\)?
	\item Let \(\mathbb{R}\) denote the set of real numbers, and \(\mathbb{R}^2\) denote the set of all points in the plane. Does there exist a silly function \(f : \mathbb{R}^2 \rightarrow \mathbb{R}\)?
	\end{enumerate}
	
	\vfill
	
	\noindent \makebox[\linewidth]{\rule{\textwidth}{0.4pt}}	
	
	\noindent \textit{Mathematical Olympiads Discord Server Invite Link:} \texttt{https://discord.gg/94UnnAG}
		
		
		
		
	
	
\end{document}