\documentclass[11pt]{article}
\usepackage[utf8]{inputenc}
\usepackage{amsmath,amsfonts,amssymb}
\usepackage[amsthm]{ntheorem} %amsthm but also no newline in lists!
\usepackage[a4paper,margin=2.5cm]{geometry} %greater control over layout
\usepackage{tikz, pgf, pgfplots} %allows tikzpictures
\pgfplotsset{compat=1.15}
\usepackage{mathrsfs}
\usetikzlibrary{arrows}
\usetikzlibrary{matrix}
\usepackage{url} %urls easy
\usepackage[shortlabels]{enumitem} %allows greater customisability of enum
\usepackage{textcomp} %gets rid of not defining \perthousand and \micro somehow
\usepackage{gensymb} %symbol package
\usepackage{sectsty}
%\usepackage{tabto} %absolute/relative horizontal jumps in text
\usepackage{perpage} %footnote counter resets each page
\usepackage[symbol]{footmisc} %use symbols instead of numbers for footnotes
\usepackage{float} %allows "H" parameter for images
\usepackage{soul} %strikethrough text available

\begin{document}
	\pagenumbering{gobble}
		\noindent \Large\textbf{Mathematical Olympiads Discord Server}
		\vspace{5pt}\\
		\noindent \huge\textbf{2019 July Beginner Contest}\\
		\noindent \makebox[\linewidth]{\rule{\textwidth}{0.4pt}}
			
	\normalsize
	
	\begin{flushright}
	\textit{Time: 4 hours} \hfill \textit{Each problem is worth 7 points}
	\end{flushright}
	
	\noindent \textit{Calculators and protractors are not allowed. Do not write your name on your working. At the end of the contest, please scan your solutions and working and send them to }\texttt{Tony Wang\#6285}\textit{ via direct message. Do not discuss the contents of this paper outside the text channel }\texttt{\#finished- contestants}\textit{ and the voice channel }\texttt{Post-Contest Banter}\textit{ until notified by staff.}
	
	\paragraph{Problem 1.} Amy writes a 1 or a \(-1\) in each cell of an \(n \times 2\) grid, where \(n\) is not a multiple of 4. Below each column she writes the product of the 2 entries in that column in blue, and to the right of each row she writes the product of the \(n\) entries in that row in green. Amy notices that the sum of the blue numbers is 0. Prove that the sum of the green numbers is 0.

	\paragraph{Problem 2.} Let \(ABCD\) be a parallelogram. The internal angle bisectors of \(ABC\) and \(BCD\) meet at \(P\), and the internal angle bisectors of \(CDA\) and \(DAB\) meet at \(Q\). Prove that \(PQ \parallel AB\).

	\paragraph{Problem 3.} Determine all positive integers that can be written as the sum of 2 or more consecutive positive integers.

	\paragraph{Problem 4.} Let \(a_1, a_2, \dots, a_n\) and \(b_1, b_2, \dots, b_n\) be positive real numbers with the property that \(a_1+a_2+ \cdots + a_n = b_1+b_2+ \cdots + b_n\). Prove that \[\sum_{i=1}^n \frac{a_i^2}{a_i+b_i} \geq \frac 12 \sum_{i=1}^n a_i.\]
	
	\vfill
	
	\noindent \makebox[\linewidth]{\rule{\textwidth}{0.4pt}}	
	
	\noindent \textit{Mathematical Olympiads Discord Server Invite Link:} \texttt{https://discord.gg/94UnnAG}
		
		
		
		
	
	
\end{document}



















